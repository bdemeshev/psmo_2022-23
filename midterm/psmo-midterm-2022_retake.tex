% arara: xelatex
\documentclass[12pt]{article}

\usepackage{physics}


\usepackage{tikz} % картинки в tikz
\usepackage{microtype} % свешивание пунктуации

\usepackage{array} % для столбцов фиксированной ширины

\usepackage{indentfirst} % отступ в первом параграфе

\usepackage{sectsty} % для центрирования названий частей
\allsectionsfont{\centering}

\usepackage{amsmath, amsfonts, amssymb} % куча стандартных математических плюшек

\usepackage{comment}

\usepackage[top=2cm, left=1.2cm, right=1.2cm, bottom=2cm]{geometry} % размер текста на странице

\usepackage{lastpage} % чтобы узнать номер последней страницы

\usepackage{enumitem} % дополнительные плюшки для списков
%  например \begin{enumerate}[resume] позволяет продолжить нумерацию в новом списке
\usepackage{caption}

\usepackage{url} % to use \url{link to web}

\usepackage{fancyhdr} % весёлые колонтитулы
\pagestyle{fancy}
\lhead{ПСМО, осень-зима 2022}
\chead{}
\rhead{Контрошечка для избранных}
\lfoot{}
\cfoot{DON'T PANIC}
\rfoot{\thepage/\pageref{LastPage}}
\renewcommand{\headrulewidth}{0.4pt}
\renewcommand{\footrulewidth}{0.4pt}

\usepackage{tcolorbox} % рамочки!

\usepackage{todonotes} % для вставки в документ заметок о том, что осталось сделать
% \todo{Здесь надо коэффициенты исправить}
% \missingfigure{Здесь будет Последний день Помпеи}
% \listoftodos - печатает все поставленные \todo'шки


% более красивые таблицы
\usepackage{booktabs}
% заповеди из докупентации:
% 1. Не используйте вертикальные линни
% 2. Не используйте двойные линии
% 3. Единицы измерения - в шапку таблицы
% 4. Не сокращайте .1 вместо 0.1
% 5. Повторяющееся значение повторяйте, а не говорите "то же"



\usepackage{fontspec}
\usepackage{polyglossia}

\setmainlanguage{russian}
\setotherlanguages{english}

% download "Linux Libertine" fonts:
% http://www.linuxlibertine.org/index.php?id=91&L=1
\setmainfont{Linux Libertine O} % or Helvetica, Arial, Cambria
% why do we need \newfontfamily:
% http://tex.stackexchange.com/questions/91507/
\newfontfamily{\cyrillicfonttt}{Linux Libertine O}

\AddEnumerateCounter{\asbuk}{\russian@alph}{щ} % для списков с русскими буквами
\setlist[enumerate, 2]{label=\asbuk*),ref=\asbuk*}

%% эконометрические сокращения
\DeclareMathOperator{\Cov}{\mathbb{C}ov}
\DeclareMathOperator{\Corr}{\mathbb{C}orr}
\DeclareMathOperator{\Var}{\mathbb{V}ar}

\let\P\relax
\DeclareMathOperator{\P}{\mathbb{P}}

\DeclareMathOperator{\E}{\mathbb{E}}
% \DeclareMathOperator{\tr}{trace}
\DeclareMathOperator{\card}{card}
\DeclareMathOperator{\plim}{plim}
\DeclareMathOperator{\pCorr}{\mathrm{p}\mathbb{C}\mathrm{orr}}


\newcommand \hb{\hat{\beta}}
\newcommand \hs{\hat{\sigma}}
\newcommand \htheta{\hat{\theta}}
\newcommand \s{\sigma}
\newcommand \hy{\hat{y}}
\newcommand \hY{\hat{Y}}
\newcommand \e{\varepsilon}
\newcommand \he{\hat{\e}}
\newcommand \z{z}
\newcommand \hVar{\widehat{\Var}}
\newcommand \hCorr{\widehat{\Corr}}
\newcommand \hCov{\widehat{\Cov}}
\newcommand \cN{\mathcal{N}}
\newcommand \RR{\mathbb{R}}
\newcommand \NN{\mathbb{N}}
\newcommand{\cF}{\mathcal{F}}
\newcommand{\cH}{\mathcal{H}}


\begin{document}


% Правила: 80 минут, можно использовать чит-лист А4 с любым содержимым: очно (заочно для тех, кто не может очно).

\begin{enumerate}

\item У Василия есть дискретная случайная величина, принимающая различные натуральные значения с некоторыми вероятностями. 
Василий заменяет каждое значение $x$, выпадающее с положительной вероятностью $p$, 
на два значения: $x-\sqrt{2}$ и $x+\sqrt{2}$, выпадающие с вероятностями $p/2$ и $p/2$, и получает новую случайную величину. 

Как при этом изменится энтропия?


\item Пусть $a$ — скалярный параметр и $I_A(a)$ — теоретическая информация Фишера.
Для  перепараметризации $a=f(b)$ найдите формулу информации Фишера $I_B(b)$.

\item Контрольную проверяют три семинариста, а за контрольную можно получить либо зачёт, либо незачёт. 
Данные по количеству работ выписаны в таблице:

\begin{tabular}{cccc}
& Андрей & Борис & Вова \\
зачёт & 30 & 40 & 50 \\
незачёт & 40 & 40 & 40 \\     
\end{tabular}

\begin{enumerate}
    \item Постройте 95\%-ый доверительный интервал для разницы вероятности получить зачёт у Андрея и Бориса. 
    \item С помощью $LR$-теста проверьте гипотезу, что вероятность получить зачёт одинакова по всем семинаристам 
    против альтернативной гипотезе хотя бы об одном отличии. 
\end{enumerate}

Табличное: правые 5\% критические значения: $\chi^2_1 = 3.84$, $\chi^2_2 = 5.99$, $\chi^2_3 = 7.81$, $\chi^2_4=9.49$.

\item Винни-Пух одновременно проверяет $h$ нулевых гипотез по независимым наборам данных. 
Все нулевые гипотезы верны, но Винни-Пух об этом не знает.

\begin{enumerate}
    \item Какова вероятность того, что наибольшее из $P$-значений окажется больше $0.95$?
    \item Как изменится вероятность из предыдущего пункта, если часть исходных нулевых гипотез не верны?
\end{enumerate}


\item 
В каждом дупле водится ровно по 10 пчёл, либо все пчелы в дупле — правильные ($b_i = \text{good}$), 
либо все — неправильные ($b_i = \text{bad}$). 
Каждая пчела приносит либо 0 килограмм мёда, либо 1 килограмм. 
У каждой правильной пчелы ожидаемое количество мёда равно $a>0.5$, а у каждой неправильной — ровно $0.5$ килограмм. 
Винни-Пух осмотрел $n$ дупел. 

Неизвестную вероятность того, что в дупле водятся правильные пчёлы, обозначим буквой $\pi$. 

Явно выпишите целевую функцию для $M$-шага $EM$-алгоритма в этой задаче, 
поясните по каким параметрам она оптимизируется и смысл остальных параметров.

\end{enumerate}



\end{document}
\newpage

\begin{enumerate}




    \item Величина $X$ принимает три возможных значения: 1, 2 и 3. 
    Вероятность $q$ приписывает им значения $0.2$, $0.3$ и $0.5$ соответственно. 
    
    \begin{enumerate}
        \item Найдите вероятность $p$, которая максимизирует дивергенцию $KL(q||p)$.
        \item Найдите вероятность $p$, которая максимизирует дивергенцию $KL(p||q)$.
    \end{enumerate}
    
Комментарий: в условии чётко написано, что других значений у $X$ вообще нет, 
но трактовка, что якобы они есть, а вероятность $q$ приписывает им нулевые веса, также будет засчитана.

\item Илон Маск максимизирует свой долгосрочный доход. 
Каждый день он играет в приятную лотерею, где с вероятностью $0.3$ ставка будет увеличена в три раза,
с вероятностью $0.2$ — в два раза, а с вероятностью $0.5$ полностью сгорит. 

\begin{enumerate}
    \item Какой процент своего благосостояния ему следует ставить каждый день?
    \item Чему равна оптимальная долгосрочная дневная доходность?
\end{enumerate}

\item В банке 10 независимых клиентских «окошек». В момент открытия в банк вошло 10 человек. Каждый
клиент встал к отдельному окошку. Других клиентов банке в этот день не было. 
Время обслуживания одного клиента распределено экспоненциально с параметром $\lambda$ 
и независимо между окошками.  

Оцените параметр $\lambda$ методом максимального правдоподобия и оцените дисперсию полученной оценки параметра в каждой из ситуаций:
\begin{enumerate}
    \item Менеджер наблюдал за окошками в течение получаса и записывал время обслуживания клиента.
    Окошко-1 обслужило своего клиента за 10 минут, окошко-2 обслужило своего клиента за 20
    минут, остальные окошки еще обслуживали своих первых клиентов в тот момент, когда менеджер
    удалился.
    \item Клиент из окошка-10 обратил внимание, что окошко-1 обслужило своего клиента за 10 минут, 
    окошко-2 обслужило своего клиента за 20
    минут, а остальные окошки еще обслуживали своих первых клиентов в тот момент, когда клиент из окошка-10 ушел по окончании своего обслуживания.
\end{enumerate}

Подсказка: при невозможности явного решения можно максимизировать полученную функцию правдоподобия на компьютере численно. 



\newpage


\item Винни-Пух стоит на пороге сенсационного открытия. Вполне возможно, что вид дерева влияет на правильность пчёл. 
Он собрал наблюдения за 200 дней в одну табличку, в каждой клетке указано количество дней. 
События, происходящие в разные дни, независимы. 


    \begin{tabular}{@{}llll@{}}
    \toprule
    & Дуб & Берёза & Ёлка \\ \midrule
    Правильные & 80 &  30  & 20\\
    Неправильные & 30  & 20 & 20 \\ \bottomrule
    \end{tabular}


\begin{enumerate}
    \item Сформулируйте гипотезу о том, что вид дерева не связан с правильностью пчёл с помощью уравнений. 
    Проверьте эту гипотезу с помощью LR, LM и теста Вальда на уровне значимости $5\%$.
    \item Постройте $95\%$ доверительный интервал для разницы доли правильных пчёл на берёзах и дубах. 
\end{enumerate}


\item Винни-Пух \textit{точно знает}, что среди осмотренных им за неделю 7 деревьев с мёдом, \textit{ровно} 5 деревьев были с правильными 
пчёлами, и \textit{ровно} 2 дерева — с неправильными. 
Логарифм количества меда для хороших и плохих пчёл распределен нормально, $\cN(\mu_g, 1)$ и $\cN(\mu_b, 1)$, соответственно. 

\begin{enumerate}
    \item Аккуратно опишите ЕМ-алгоритм для данной ситуации.
    \item В чем отличие от ситуации с независимыми наблюдениями?
\end{enumerate}



\item Риши Сунак по вступлении в должность сразу принялся одновременно тестировать четыре верных гипотезы $H_0^i$ по независимым наборам данным.
Он поставил в качестве цели $FWER \leq 0.05$ и использует для реализации две поправки: Бонферонни и Холма-Бонферонни. 

Каким будет фактический $FWER$ в каждом из этих случаев?

\item {[бонус]} Здесь можно написать всё, что угодно. 
Поделиться классной новой задачкой, книжкой или фильмом, пожаловаться на что-то, никаких ограничений!

\end{enumerate}


\end{document}

