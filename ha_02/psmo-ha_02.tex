% arara: xelatex
\documentclass[12pt]{article}

\usepackage{physics}


\usepackage{tikz} % картинки в tikz
\usepackage{microtype} % свешивание пунктуации

\usepackage{array} % для столбцов фиксированной ширины

\usepackage{indentfirst} % отступ в первом параграфе

\usepackage{sectsty} % для центрирования названий частей
\allsectionsfont{\centering}

\usepackage{amsmath, amsfonts, amssymb} % куча стандартных математических плюшек

\usepackage{comment}

\usepackage[top=2cm, left=1.2cm, right=1.2cm, bottom=2cm]{geometry} % размер текста на странице

\usepackage{lastpage} % чтобы узнать номер последней страницы

\usepackage{enumitem} % дополнительные плюшки для списков
%  например \begin{enumerate}[resume] позволяет продолжить нумерацию в новом списке
\usepackage{caption}

\usepackage{url} % to use \url{link to web}

\usepackage{fancyhdr} % весёлые колонтитулы
\pagestyle{fancy}
\lhead{ПСМО, осень-зима 2022}
\chead{}
\rhead{Домяшка-02, 2022-12-04, 21:00}
\lfoot{$^1$ Целевая функция $\sum_{i=1}^n \rho_{\tau}(y_i - \hat y_i)$, где $\rho_{\tau}(u) = u\cdot (\tau - I(u<0))$.}
\cfoot{}
\rfoot{\thepage/\pageref{LastPage}}
\renewcommand{\headrulewidth}{0.4pt}
\renewcommand{\footrulewidth}{0.4pt}

\usepackage{tcolorbox} % рамочки!

\usepackage{todonotes} % для вставки в документ заметок о том, что осталось сделать
% \todo{Здесь надо коэффициенты исправить}
% \missingfigure{Здесь будет Последний день Помпеи}
% \listoftodos - печатает все поставленные \todo'шки


% более красивые таблицы
\usepackage{booktabs}
% заповеди из докупентации:
% 1. Не используйте вертикальные линни
% 2. Не используйте двойные линии
% 3. Единицы измерения - в шапку таблицы
% 4. Не сокращайте .1 вместо 0.1
% 5. Повторяющееся значение повторяйте, а не говорите "то же"



\usepackage{fontspec}
\usepackage{polyglossia}

\setmainlanguage{russian}
\setotherlanguages{english}

% download "Linux Libertine" fonts:
% http://www.linuxlibertine.org/index.php?id=91&L=1
\setmainfont{Linux Libertine O} % or Helvetica, Arial, Cambria
% why do we need \newfontfamily:
% http://tex.stackexchange.com/questions/91507/
\newfontfamily{\cyrillicfonttt}{Linux Libertine O}

\AddEnumerateCounter{\asbuk}{\russian@alph}{щ} % для списков с русскими буквами
\setlist[enumerate, 2]{label=\asbuk*),ref=\asbuk*}

%% эконометрические сокращения
\DeclareMathOperator{\Cov}{\mathbb{C}ov}
\DeclareMathOperator{\Corr}{\mathbb{C}orr}
\DeclareMathOperator{\Var}{\mathbb{V}ar}

\let\P\relax
\DeclareMathOperator{\P}{\mathbb{P}}

\DeclareMathOperator{\E}{\mathbb{E}}
% \DeclareMathOperator{\tr}{trace}
\DeclareMathOperator{\card}{card}
\DeclareMathOperator{\plim}{plim}
\DeclareMathOperator{\pCorr}{\mathrm{p}\mathbb{C}\mathrm{orr}}


\newcommand \hb{\hat{\beta}}
\newcommand \hs{\hat{\sigma}}
\newcommand \htheta{\hat{\theta}}
\newcommand \s{\sigma}
\newcommand \hy{\hat{y}}
\newcommand \hY{\hat{Y}}
\newcommand \e{\varepsilon}
\newcommand \he{\hat{\e}}
\newcommand \z{z}
\newcommand \hVar{\widehat{\Var}}
\newcommand \hCorr{\widehat{\Corr}}
\newcommand \hCov{\widehat{\Cov}}
\newcommand \cN{\mathcal{N}}
\newcommand \RR{\mathbb{R}}
\newcommand \NN{\mathbb{N}}
\newcommand{\cF}{\mathcal{F}}
\newcommand{\cH}{\mathcal{H}}


\begin{document}


Дедлайн: 2022-12-04, \textbf{21:00}. Если вычислительные возможности позволяют, берите $n_{boot}=10000$ бутстрэп-выборок. 
Если нет, то сокращайте в соответствии с вычислительными возможностями. 

\begin{enumerate}

\item Рассмотрим наивный бутстрэп для исходной выборки из $n$ наблюдений непрерывного распределения. 
Обозначим исходные наблюдения с помощью $y_i$, а бутстрэпированные — с помощью $y_i^*$.

\begin{enumerate}
    \item Какова вероятность того, что пятое исходное наблюдение ровно три раза попадет в бутстрэп выборку?        
    \item Чему равен предел найденной в (а) вероятности при $n\to \infty$?
    \item Чему равен предел вероятности $\P(\max y_i > \max y_i^*)$ при $n\to \infty$?
\end{enumerate}




\item Величины $y_i$ независимы $\cN(\mu;1)$. А-приори известно, что $\mu \geq 42$.
\begin{enumerate}
    \item Без компьютера явно найдите оценку $\mu$  методом максимального правдоподобия.
\end{enumerate}
Винни-Пух использует наивный бутстрэп для построения 95\%-го  интервала для $\mu$. 
\begin{enumerate}[resume]
    \item С помощью симуляций оцените фактическую вероятность накрытия, если истинное $\mu = 42$.
    \item С помощью симуляций оцените фактическую вероятность накрытия, если истинное $\mu = 44$.
\end{enumerate}


\item Величины $y_i$ независимы $\cN(0;1)$. 
Кролик не знает истинное ожидание, и для надежности использует \text{очень хитрый} способ оценки $\mu$. 
В качестве $\hat \mu$ Кролик выбирает середину самого короткого отрезка, содержащего 50\% выборки. 
Для построения 95\%-го интервала для $\mu$ Кролик использует наивный бутстрэп. 

С помощью симуляций оцените фактическую вероятность накрытия для $n=100$.


\end{enumerate}



Следующие вопросы относятся к набору данных \url{https://github.com/bdemeshev/psmo_2022-23/raw/main/ha_02/psmo-ha_02.csv}. 

\begin{enumerate}[resume]
    \item Постройте 95\%-й доверительный интервал для медианы величины $y$, используя наивный бутстрэп, бутстрэп $t$-статистики и BCA-бутстрэп. 

    \item Рассмотрим модель линейной регрессии 
    \[
    \hat y_i = \hat \beta_1 + \hat\beta_2 x_i + \hat\beta_3 x_i^2.
    \]
    \begin{enumerate}
    \item С помощью наивного бутстрэпа постройте 95\%-й доверительный интервал для каждого коэффициента.
    \item С помощью наивного бутстрэпа постройте 95\%-й доверительный интервал для вершины параболы. 
    \end{enumerate}
\item Рассмотрим модель квантильной регрессии$^1$ для квантилей $\tau=0.1$, $\tau=0.5$, $\tau=0.9$:
    \[
    \hat y_i = \hat \beta_1 + \hat\beta_2 x_i + \hat\beta_3 x_i^2.
    \]
    \begin{enumerate}
    \item Постройте три полученные линии регрессии на одном графике с данными. 
    \item Для медианной регрессии с помощью наивного бутстрэпа постройте 95\%-й доверительный интервал для каждого коэффициента.
    \item Для медианной регрессии с помощью наивного бутстрэпа постройте 95\%-й доверительный интервал для вершины параболы. 
    \end{enumerate}


\end{enumerate}




\end{document}
