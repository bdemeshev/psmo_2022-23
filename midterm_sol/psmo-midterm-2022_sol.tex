% arara: xelatex
\documentclass[12pt]{article}

\usepackage{physics}


\usepackage{tikz} % картинки в tikz
\usepackage{microtype} % свешивание пунктуации

\usepackage{array} % для столбцов фиксированной ширины

\usepackage{indentfirst} % отступ в первом параграфе

\usepackage{sectsty} % для центрирования названий частей
\allsectionsfont{\centering}

\usepackage{amsmath, amsfonts, amssymb} % куча стандартных математических плюшек

\usepackage{comment}

\usepackage[top=2cm, left=1.2cm, right=1.2cm, bottom=2cm]{geometry} % размер текста на странице

\usepackage{lastpage} % чтобы узнать номер последней страницы

\usepackage{enumitem} % дополнительные плюшки для списков
%  например \begin{enumerate}[resume] позволяет продолжить нумерацию в новом списке
\usepackage{caption}

\usepackage{url} % to use \url{link to web}

\usepackage{fancyhdr} % весёлые колонтитулы
\pagestyle{fancy}
\lhead{ПСМО, осень-зима 2022}
\chead{}
\rhead{Контрошечка, 2022-11-18}
\lfoot{}
\cfoot{DON'T PANIC}
\rfoot{\thepage/\pageref{LastPage}}
\renewcommand{\headrulewidth}{0.4pt}
\renewcommand{\footrulewidth}{0.4pt}

\usepackage{tcolorbox} % рамочки!

\usepackage{todonotes} % для вставки в документ заметок о том, что осталось сделать
% \todo{Здесь надо коэффициенты исправить}
% \missingfigure{Здесь будет Последний день Помпеи}
% \listoftodos - печатает все поставленные \todo'шки


% более красивые таблицы
\usepackage{booktabs}
% заповеди из докупентации:
% 1. Не используйте вертикальные линни
% 2. Не используйте двойные линии
% 3. Единицы измерения - в шапку таблицы
% 4. Не сокращайте .1 вместо 0.1
% 5. Повторяющееся значение повторяйте, а не говорите "то же"



\usepackage{fontspec}
\usepackage{polyglossia}

\setmainlanguage{russian}
\setotherlanguages{english}

% download "Linux Libertine" fonts:
% http://www.linuxlibertine.org/index.php?id=91&L=1
\setmainfont{Linux Libertine O} % or Helvetica, Arial, Cambria
% why do we need \newfontfamily:
% http://tex.stackexchange.com/questions/91507/
\newfontfamily{\cyrillicfonttt}{Linux Libertine O}

\AddEnumerateCounter{\asbuk}{\russian@alph}{щ} % для списков с русскими буквами
\setlist[enumerate, 2]{label=\asbuk*),ref=\asbuk*}

%% эконометрические сокращения
\DeclareMathOperator{\Cov}{\mathbb{C}ov}
\DeclareMathOperator{\Corr}{\mathbb{C}orr}
\DeclareMathOperator{\Var}{\mathbb{V}ar}

\let\P\relax
\DeclareMathOperator{\P}{\mathbb{P}}

\DeclareMathOperator{\E}{\mathbb{E}}
% \DeclareMathOperator{\tr}{trace}
\DeclareMathOperator{\card}{card}
\DeclareMathOperator{\plim}{plim}
\DeclareMathOperator{\pCorr}{\mathrm{p}\mathbb{C}\mathrm{orr}}


\newcommand \hb{\hat{\beta}}
\newcommand \hs{\hat{\sigma}}
\newcommand \htheta{\hat{\theta}}
\newcommand \s{\sigma}
\newcommand \hy{\hat{y}}
\newcommand \hY{\hat{Y}}
\newcommand \e{\varepsilon}
\newcommand \he{\hat{\e}}
\newcommand \z{z}
\newcommand \hVar{\widehat{\Var}}
\newcommand \hCorr{\widehat{\Corr}}
\newcommand \hCov{\widehat{\Cov}}
\newcommand \cN{\mathcal{N}}
\newcommand \RR{\mathbb{R}}
\newcommand \NN{\mathbb{N}}
\newcommand{\cF}{\mathcal{F}}
\newcommand{\cH}{\mathcal{H}}


\begin{document}


% Правила: 80 минут, можно использовать чит-лист А4 с любым содержимым: очно (заочно для тех, кто не может очно).

Общее: Каждая задача оценивается целым числом баллов от 0 до 6.
Все результаты масштабируются в процентах от второй наилучшей работы. 


\begin{enumerate}

\item Ёжику досталась обучающая выборка для задачи классификации: \newline
целевая переменная $y=(1, -1, -1, 1, 1)$ и предиктор $x=(1, 2, 3, 4, 5)$.

Медвежонок выбирает одно наблюдение из пяти равновероятно, обозначим с помощью $Y$ и $X$ 
полученные случайные значения целевой переменной и предиктора. 
Ёжик может задать Медвежонку один вопрос вида «правда ли, что $X$ больше константы $c$?».

Какую $c$ надо выбрать Ёжику, чтобы минимизировать $H(Y|Q)$, где $Q$ — это ответ Медвежонка? 


Подсказка: если $h(t) = -t\ln t$, то $h(1/2)=0.347$, $h(1/3)=0.366$, $h(2/3)=0.270$, $h(3/4)=0.216$, $h(2/5)=0.367$,
впрочем, задачу можно полностью решить и без этих цифр :)

Решение:

\[
H(Y|Q) = \P(Q=\text{да}) H(Y|Q=\text{да}) + \P(Q=\text{нет}) H(Y|Q=\text{нет})    
\]

Обозначим $a(p)$ — энтропию дискретной величины, принимающей значения 1 и (-1) с вероятностями $p$ и $1-p$,
сразу замечаем, что $a(0) = a(1) = 0$, и чем ближе $p$ к $1/2$, тем больше $a(p)$, $a(p)=a(1-p)$.

Делаем быстрый перебор:

$c = 0$ (всё в одну кучу):
\[
H(Y|Q) = a(0.4)    
\]
$c=1.5$:
\[
H(Y|Q) = 0.2 a(1) + 0.8 a(1/2) = 0.8 a(1/2)
\]
$c=2.5$:
\[
H(Y|Q) = 0.4 a(1/2) + 0.6 a(1/3) 
\]
$c=3.5$:
\[
H(Y|Q) = 0.6 a(1/3) + 0.4 a(1) = 0.6a(1/3) 
\]
$c=4.5$:
\[
H(Y|Q) = 0.8 a(1/2) + 0.2 a(1) = 0.8a(1/2) 
\]

Замечаем, что $a(1/3)<a(0.4)<a(1/2)$, можно устно, можно и числа подставить.

Следовательно, $0.6a(1/3)$ меньше, чем $a(0.4)$, $0.8a(1/2)$ или $0.4a(1/2)+0.6a(1/3)$.

Все $c\in[3;4)$ подойдут.


Верно применена формула для условной энтропии хотя бы один раз: +3 балла. 

Аккуратный перебор всех случаев: + 3 балла. 

Если угадано $c=3$ с неверным объяснением: +1 за удачливость. 


\item Ежу понятно, что математическое ожидание первой производной лог-функции правдоподобия тождественно равно нулю.
Производная нуля равна нулю. Для хорошей функции правдопобия производная от ожидания первой производной равна ожиданию второй производной. 
Ожидание второй производной лог-функции правдоподобия со знаком минус называется информацией Фишера. 
Следовательно, информация Фишера всегда равна нулю. 

Найдите качественную ошибку в этом рассуждении. 


Решение:

Ожидание производной лог-правдоподобия зависит от двух аргументов:
истинного параметра, по которому считается ожидание, и аргумента лог функция правдоподобия, от которой считается ожидание. 

\[
g(\theta, \theta^{T}) = \int_{\mathbb{R}^n} f(y \mid \theta^T) \frac{\partial }{\partial \theta}\ln f(y \mid \theta) dy.
\]

Производная равна нулю только в точке, где аргумент функции лог-правдоподобия совпадает с истинным параметром, 
\[
g(\theta^T, \theta^T) = 0    
\]
Равенство нулю производной в точке не означает равенства нулю следующей производной. Всё.
\[
\frac{\partial}{\partial \theta} g(\theta, \theta^{T}) \neq 0 \text{ в точке } \theta = \theta^T.   
\]

Достаточно текстового объяснения без формул. 

Если угадано, в каком логическом переходе ошибка, но не объяснено, какая именно: +1. 

Упомянута регулярность: +1.

Часто говорили, что у функции правдоподобия оптимум там, где производная равна нулю, но в задаче речь об ожидании правдоподобия. 


\item Хорошо обученная свинья тратит на поиск одного трюфеля экспоненциальное время с ожиданием $\mu$ минут. 
Поиск различных трюфелей независим. 
Жерар Депардье, к сожалению, замерял время поиска 100 трюфелей только тремя диапазонами: 
от 0 до 10 минут (20 трюфелей), от 10 до 20 минут (50 трюфелей), более 20 минут (30 трюфелей). 

\begin{enumerate}
    \item Постройте 95\% асимптотический доверительный интервал для $\mu$. 
    \item С помощью $LM$-теста проверьте гипотезу о том, что $\mu = 10$ против альтернативной 
    гипотезы о неравенстве для уровня значимости 5\%.
\end{enumerate}

Табличное: правые 5\% критические значения: $\chi^2_1 = 3.84$, $\chi^2_2 = 5.99$, $\chi^2_3 = 7.81$, $\chi^2_4=9.49$,
функция плотности экспоненциального распределения имеет вид $\lambda \exp(-\lambda x)$.

Решение:

Для экспоненциального распределения вероятность попасть от $a$ до $b$ равна $\exp(-a/\mu) - \exp(-b/\mu)$.

В нашем случае: $\P(Y_i < 10) = (1 - \exp(-10/\mu)) = 1 - t$, $\P(Y_i \in [10;20]) = (\exp(-20/\mu) - \exp(-10/\mu)) = t(1-t)$,
$\P(Y_i > 20) = \exp(-20/\mu) = t^2$. 

Функция правдоподобия равна
\[
L = \frac{100!}{20!30!50!} (1-t)^{20} (t(1-t))^{50}(t^2)^{30} = c t^{110}(1-t)^{70}
\]

Экстремум равен $\hat t = 110/180=11/18$, $-10/\hat \mu = \ln 11 - \ln 18$, 


Выписана функция правдоподобия: 1 балл.

Найдена оценка: 1 балл.

Найдена стандартная ошибка: 1 балл.

Найден доверительный интервал: 1 балла. 

Проведен LM тест: 2 балла. 


\item Ёж проверяет всего две гипотезы $H_0^i$ одновременно, одна из которых верна, а вторая — нет. 
Для неверной гипотезы $P$-значение распределено равномерно на $[0;0.5]$.

Ёж сортирует $P$-значения по возрастанию, $p_{(1)} \leq p_{(2)}$. Затем сравнивает $p_{(1)}$ с константой $0.05$.
Если $p_{(1)} \geq 0.05$, то обе $H_0^{(i)}$ не отвергаются и Ёж заканчивает работу. 
Если $p_{(1)} < 0.05$, то $H_0^{(1)}$ отвергается и Ёж сравнивает $p_{(2)}$ с константой $b$. 
Если $p_{(2)} < b$, то $H_0^{(2)}$ отвергается, иначе $H_0^{(2)}$ не отвергается.

Постройте график зависимости $FDR$ (false discovery rate) от $b \in [0.05;1]$.

Пусть $N_R$ — число отвергнутых нулевых гипотез, а $N_{WR}$ — число ошибочно отвергнутых нулевых гипотез.


У нас всего-то три случая:
\[
FDR = \E\left(\frac{N_{WR}}{\max\{N_R, 1\}} \right)  =  \sum_{i=0}^2\E\left(\frac{N_{WR}}{\max\{N_R, 1\}} \cdot I(N_{R} = i)\right) 
\]

Находим каждое слагаемое. 
Первое:
\[
\E\left(\frac{N_{WR}}{\max\{N_R, 1\}} \cdot I(N_{R} = 2)\right) = 0.5 \P(N_R = 2) 
\]
Находим вероятность того, что обе гипотезы отвергнуты.
Это вероятность того, что оба $P$-значения меньше $b$ и хотя бы одно из них меньше $0.05$.
Можно закрасить соответствующую фигуру на прямоугольнике, где совместная плотность больше нуля, 
и помножить площадь на плотность, получится:
\[
    \P(N_R = 2)  = \text{площадь} \cdot 2 = (2\cdot 0.05b - 0.05^2) \cdot 2.
\]
Второе:
\[
    \E\left(\frac{N_{WR}}{\max\{N_R, 1\}} \cdot I(N_{R} = 1)\right) = \P(N_{WR} = 1, N_{R} = 1)
\]
Находим вероятность того, что отвергнута только та гипотеза, что была верна.
\[
\P(N_{WR} = 1, N_{R} = 1) = \P(p_T < 0.05, p_F > b) =\P(p_T < 0.05)\P(p_F > b) = 0.05 \cdot (1 - 2b).
\]
Третье:
\[
    \E\left(\frac{N_{WR}}{\max\{N_R, 1\}} \cdot I(N_{R} = 0)\right) = 0
\]


Выписана $FDR$ в верном виде применительно к этой задаче (видно, что будет три случая): +3

Рассмотрен каждый случай: +1 за случай. 



\item 
Пчёлы бывают правильные ($b_i = \text{good}$) и неправильные ($b_i = \text{bad}$). 
Из одного дупла правильных пчёл можно извлечь случайное равномерное количество мёда, $(y_i \mid b_i = \text{good}) \sim U[0;a]$, 
где параметр $a$ не известен и $a>1$.
Для одного дупла неправильных $(y_i \mid b_i = \text{bad}) \sim U[0;1]$.
Имеется $n$ независимых наблюдений. 
Неизвестную вероятность того, что в дупле водятся правильные пчёлы, обозначим буквой $\pi$. 

Явно выпишите целевую функцию для $M$-шага $EM$-алгоритма в этой задаче, поясните по каким параметрам она оптимизируется и смысл остальных параметров.


Решение:

Обозначим $g_i = \P(b_i = \text{good} \mid a_{old}, \pi_{old})$.
В задаче не требовалось находить эту функцию, но вот она:
\[
g_i = 
\begin{cases}
1, \text{ если } y_i > 1; \\
\frac{\pi_{old}/a_{old}}{\pi_{old}/a_{old} + (1- \pi_{old})\cdot 1}, \text{ если } y_i \leq 1.
\end{cases}
\]

Целевая функция равна:
\[
Q(a, \pi \mid a_{old}, \pi_{old}) = 
\begin{cases}
\sum_{i=1}^n g_i \ln (\pi / a) + (1 - g_i) \ln (1 - \pi), \text{ если } y_i \leq 1 \text{ при всех } g_i < 1;\\     
-\infty, \text{ иначе}.
\end{cases}
\]

Максимизируем по $a$, $\pi$. 


Корректно выписано в общем виде и при этом разделены тета old от тета: 3 балла. 

Явно указано по каким переменным идет оптимизация: +1 балл.

Явно выписаны функции плотности для данной задачи: +2 балла. 

\end{enumerate}



\end{document}
